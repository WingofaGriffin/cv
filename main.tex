%%%%%%%%%%%%%%%%%
% This is an sample CV template created using altacv.cls
% (v1.1.5, 1 December 2018) written by LianTze Lim (liantze@gmail.com). Now compiles with pdfLaTeX, XeLaTeX and LuaLaTeX.
%
%% It may be distributed and/or modified under the
%% conditions of the LaTeX Project Public License, either version 1.3
%% of this license or (at your option) any later version.
%% The latest version of this license is in
%%    http://www.latex-project.org/lppl.txt
%% and version 1.3 or later is part of all distributions of LaTeX
%% version 2003/12/01 or later.
%%%%%%%%%%%%%%%%

%% If you need to pass whatever options to xcolor
\PassOptionsToPackage{dvipsnames}{xcolor}

%% If you are using \orcid or academicons
%% icons, make sure you have the academicons
%% option here, and compile with XeLaTeX
%% or LuaLaTeX.
% \documentclass[10pt,a4paper,academicons]{altacv}

%% Use the "normalphoto" option if you want a normal photo instead of cropped to a circle
% \documentclass[10pt,a4paper,normalphoto]{altacv}

\documentclass[10pt,a4paper,ragged2e]{altacv}

%% AltaCV uses the fontawesome and academicon fonts
%% and packages.
%% See texdoc.net/pkg/fontawecome and http://texdoc.net/pkg/academicons for full list of symbols. You MUST compile with XeLaTeX or LuaLaTeX if you want to use academicons.

% Change the page layout if you need to
\geometry{left=1cm,right=9cm,marginparwidth=6.8cm,marginparsep=1.25cm,top=1.25cm,bottom=1.25cm}

% Change the font if you want to, depending on whether
% you're using pdflatex or xelatex/lualatex
\ifxetexorluatex
  % If using xelatex or lualatex:
  \setmainfont{Carlito}
\else
  % If using pdflatex:
  \usepackage[utf8]{inputenc}
  \usepackage[T1]{fontenc}
  \usepackage[default]{raleway}
\fi

% Change the colours if you want to
\definecolor{DarkSlateBlue}{HTML}{483D8B}
\definecolor{MidnightBlue}{HTML}{00008B}
\definecolor{SlateGrey}{HTML}{2E2E2E}
\definecolor{LightGrey}{HTML}{666666}
\colorlet{heading}{MidnightBlue}
\colorlet{accent}{DarkSlateBlue}
\colorlet{emphasis}{SlateGrey}
\colorlet{body}{LightGrey}

% Change the bullets for itemize and rating marker
% for \cvskill if you want to
\renewcommand{\itemmarker}{{\small\textbullet}}
\renewcommand{\ratingmarker}{\faCircle}

%% sample.bib contains your publications
%\addbibresource{sample.bib}

\usepackage[none]{hyphenat}

\begin{document}
\name{Griffin Solot-Kehl}
\tagline{Developer Advocate at Transposit}
%\photo{2.8cm}{griffin}
\personalinfo{%
  % Not all of these are required!
  % You can add your own with \printinfo{symbol}{detail}
  \email{griffin.z.s@outlook.com}
  \phone{(415) 305-3193}
  \location{San Francisco, CA}
  \\
  \vspace{5pt}
  \homepage{gzs.fyi}
  \linkedin{griffinzs}
  \github{wingofagriffin}
  \twitter{wingofagriffin}
  %% You MUST add the academicons option to \documentclass, then compile with LuaLaTeX or XeLaTeX, if you want to use \orcid or other academicons commands.
  % \orcid{orcid.org/0000-0000-0000-0000}
}

%% Make the header extend all the way to the right, if you want.
%\begin{fullwidth}
\makecvheader
%\end{fullwidth}

%% Depending on your tastes, you may want to make fonts of itemize environments slightly smaller
% \AtBeginEnvironment{itemize}{\small}

%% Provide the file name containing the sidebar contents as an optional parameter to \cvsection.
%% You can always just use \marginpar{...} if you do
%% not need to align the top of the contents to any
%% \cvsection title in the "main" bar.
\cvsection[page1sidebar]{Experience}

\cvevent{Developer Advocate}{Transposit}{June 2019 -- October 2019}{San Francisco, CA}
\begin{itemize}
\item Launched sample apps and blog posts, and worked with individuals to inspire and promote adoption of platform.
\item Managed, and spoke at developer events to create relationships between users and partners.
\item Optimized documentation, and built landing pages for partner integration and collaboration.
\end{itemize}
\divider

\cvevent{Teaching Assistant}{Yale School of Management}{August 2018 -- December 2018}{New Haven, CT}
\begin{itemize}
\item Taught fundamentals of Agile Project Management and Scrum.
\item Assisted MBA candidates in building Model, View, Controller based web apps using JavaScript, HTML, and CSS.
\end{itemize}
\divider

\cvevent{Developer Evangelist Intern}{DocuSign Inc.}{May 2018 -- August 2018}{San Francisco, CA}
\begin{itemize}
\item Worked with product, engineering and marketing to find ways to improve and spread the developer on-boarding experience.
\item Organized hackathon attendance and participation to spread API specifications SDK to new users to build creative applications.
\item Designed and wrote teaching content for DocuSign's booth at Dreamforce 2018 on the use of DocuSign for Salesforce API.
\end{itemize}
\divider

\cvevent{Intern}{Mahlzeit GmbH}{May 2017 -- August 2018}{Amsterdam, NL}
\begin{itemize}
\item Marketed product to European companies in start-up accelerator.
\item Built modernized interface to show status of canteens. Large focus on UX.
\end{itemize}
\begin{comment}
\cvsection{Projects}

\cvevent{Project 1}{Funding agency/institution}{}{}
\begin{itemize}
\item Details
\end{itemize}

\divider

\cvevent{Project 2}{Funding agency/institution}{Project duration}{}
A short abstract would also work.
\end{comment}

\medskip

\cvsection{A Typical Day}

% Adapted from @Jake's answer from http://tex.stackexchange.com/a/82729/226
% \wheelchart{outer radius}{inner radius}{
% comma-separated list of value/text width/color/detail}
\wheelchart{1.5cm}{0.5cm}{%
  2/8em/accent!25/{Tinkering with Arch Linux},
  8/8em/accent/Getting sleep,
  1/6em/accent!85/Self care and Music,
  3/10em/accent!70/Cooking (and Eating),
  7/10em/accent!55/Building Developer Relationships,
  3/8em/accent!40/Socializing online and offline 
}

\clearpage

%% If the NEXT page doesn't start with a \cvsection but you'd
%% still like to add a sidebar, then use this command on THIS
%% page to add it. The optional argument lets you pull up the
%% sidebar a bit so that it looks aligned with the top of the
%% main column.
% \addnextpagesidebar[-1ex]{page3sidebar}

\end{document}
